% Document class and options
\documentclass[9pt, a5paper]{article}

% Set margins
\usepackage{geometry}
\geometry{lmargin=2cm, rmargin=2cm, tmargin=2cm, bmargin=2cm}

% Change default fonts
\usepackage{mathpazo}

% Customize ToC
\usepackage[titles]{tocloft}
\renewcommand\cftsecfont{\normalsize\itshape}
\renewcommand\cftsecpagefont{\normalsize\bfseries}
\renewcommand{\cftsecleader}{\cftdotfill{\cftdotsep}}

% Mathematics environments
\usepackage{amssymb}
\usepackage{amsmath}
\usepackage{amstext}

% Language and hyphenation
\usepackage[brazil]{babel}
\usepackage[utf8]{inputenc}
\usepackage[T1]{fontenc}

% Hyper references
\usepackage[linktocpage]{hyperref}
\hypersetup{colorlinks=true,
            urlcolor=blue,
            linkcolor=teal}

% Add blank pages
\usepackage{afterpage}
\newcommand\blankpage{%
    \null
    \thispagestyle{empty}%
    \addtocounter{page}{-1}%
    \newpage}

% Footnotes in bottom and with same number
\usepackage[bottom]{footmisc}
\usepackage{perpage}
\MakePerPage{footnote}

% Tables
\usepackage{array}
\usepackage{tabularx}
\newcolumntype{L}[1]{>{\raggedright\let\newline\\\arraybackslash\hspace{0pt}}m{#1}}

% General purposes
\usepackage{graphicx}
\usepackage{xcolor}

% Metadata
\title{\fontsize{38}{32} \bfseries Seminários Acadêmicos}
\author{Estatística e Experimentação Agronômica}
\date{\small \url{https://esalq-ppgeea.github.io/discentes/seminars.html}}

\begin{document}

\maketitle
\vspace{1cm}

\begin{center}
  \textbf{\large Edição 2018}
\end{center}

\vfill
\hspace{0.08cm}
\begin{minipage}{\textwidth}
  \small
  \begin{tabular}{L{1cm}L{3cm}L{1.42cm}L{3cm}}
    \includegraphics[height=2cm]{logo-esalq} &
    \scalebox{.8}[1.0]{Escola Superior de Agri--}\newline
    \scalebox{.8}[1.0]{cultura ``Luiz de Queiroz''}\newline
    \vspace{0.1cm}
    \scalebox{.8}[1.0]{Departamento de}\newline
    \scalebox{.8}[1.0]{Ciências Exatas} &
    % -------------------------------------------
    % -------------------------------------------
    \includegraphics[height=2cm, width=1.8cm]{logo-ppgeea} &
    \scalebox{.8}[1.0]{Programa de Pós-gra--}\newline
    \scalebox{.8}[1.0]{duação em Estatística}\newline
    \scalebox{.8}[1.0]{e Experimentação}\newline
    \scalebox{.8}[1.0]{Agronômica}
  \end{tabular}
\end{minipage}


\thispagestyle{empty}
\afterpage{\blankpage}
\clearpage

%-------------------------------------------
\tableofcontents
\thispagestyle{empty}
\clearpage

%=======================================================================
% Date, Authors, Title and Abstracts
%=======================================================================

%-----------------------------------------------------------------------
% 17/05
%-----------------------------------------------------------------------
\addcontentsline{toc}{section}{(17/05)
  Uso do Shiny para o ensino-aprendizagem de Inferência
  Bayesiana}

\begin{center}
  {\footnotesize 17/05/2018 (Quinta-feira)}\\
  \textbf{\large
    Uso do Shiny para o ensino-aprendizagem de
    Inferência Bayesiana}\\
  Cristian Villegas\footnote{ESALQ/USP, contato:
    \url{jreduardo@usp.br}},
  Eduardo Ribeiro Jr\footnotemark[1] \&
  Roseli Leandro\footnotemark[1]
\end{center}

 Com o aumento da complexidade dos modelos empregados na análise de
 dados, a inferência considerando-se a abordagem bayesiana tem sido cada
 vez mais utilizada em diferentes áreas de pesquisa. Com o objetivo
 facilitar o ensino-aprendizagem da estatística bayesiana, cada dia mais
 necessário, propõe-se este minicurso, que irá explorar conceitos
 básicos tais como: distribuição a priori (informativas e não
 informativas), função de verossimilhança, distribuição a posteriori, e
 técnicas de amostragem Metropolis-Hastings e amostrador de Gibbs, com a
 utilização do pacote R de código aberto R shiny
 (\url{http://www.shiny.rstudio.com}) para elaboração de interfaces
 interativas. O Shiny fornece uma estrutura elegante, poderosa e
 estimulante para criar aplicativos web sem requerer conhecimento em
 HTML, CSS ou JavaScript, propiciando a melhoria no processo de
 ensino-aprendizado. Nesse minicurso serão abordados modelos lineares,
 lineares generalizados e não lineares sob a perspectiva bayesiana, com
 aplicações envolvendo dados reais de estudos experimentais. Os recursos
 pra criação de interfaces interativas serão disponibilizados em um
 pacote R para que possam ser utilizados i) por estudantes ou
 entusiastas em estatística bayesiana, para o aprendizado; e ii) por
 docentes ou responsáveis por cursos de estatística bayesiana, como
 ferramenta complementar para o ensino.

\clearpage

%-----------------------------------------------------------------------
% 16/08
%-----------------------------------------------------------------------
\addcontentsline{toc}{section}{(16/08)
  Reparametrization of COM-Poisson Regression Models}

\begin{center}
  {\footnotesize 16/08/2018 (Quinta-feira)}\\
  \textbf{\large
    Reparametrization of COM-Poisson Regression Models}\\
  Eduardo Ribeiro Jr\footnote{ESALQ/USP, contato:
    \url{jreduardo@usp.br}}
\end{center}

In the analysis of count data often the equidispersion assumption is not
suitable, hence the Poisson regression model is inappropriate. As a
generalization of the Poisson distribution the COM-Poisson distribution
can deal with under-, equi- and overdispersed count data. It is a member
of the exponential family of distributions and has the Poisson and
geometric distributions as special cases, as well as the Bernoulli
distribution as a limiting case. In spite of the nice properties of the
COM-Poisson distribution, its location parameter does not correspond to
the expectation, which complicates the interpretation of regression
models specified using this distribution. In this paper, we propose a
straightforward reparametrization of the COM-Poisson distribution based
on an approximation to the expectation of this distribution. The main
advantage of our new parametrization is the straightforward
interpretation of the regression coefficients in terms of the
expectation of the count response variable, as usual in the context of
generalized linear models. Furthermore, the estimation and inference for
the new COM-Poisson regression model can be done based on the likelihood
paradigm. We carried out simulation studies to verify the finite sample
properties of the maximum likelihood estimators. The results from our
simulation study show that the maximum likeli-hood estimators are
unbiased and consistent for both regression and dispersion
parameters. We observed that the empirical correlation between the
regression and dispersion parameter estimators is close to zero, which
suggests that these parameters are orthogonal. We illustrate the
application of the proposed model through the analysis of three data
sets with over-, under- and equidispersed count data. The study of
distribution properties through a consideration of dispersion,
zero-inflated and heavy tail indexes, together with the results of data
analysis show the flexibility over standard approaches. Therefore, we
encourage the application of the new parametrization for the analysis of
count data in the context of COM-Poisson regression models.

\clearpage

%-----------------------------------------------------------------------
% 23/08
%-----------------------------------------------------------------------
\addcontentsline{toc}{section}{(23/08)
  Bivariate residual plots with simulation polygons}

\begin{center}
  {\footnotesize 23/08/2018 (Quinta-feira)}\\
  \textbf{\large
    Bivariate residual plots with simulation polygons}\\
  Rafael de Andrade Moral\footnote{Maynooth University, contato:
    \url{rafael.deandrademoral@mu.ie}}
\end{center}

When using univariate models, goodness-of-fit can be assessed through
many different methods, including graphical tools such as half-normal
plots with a simulation envelope. This is straightforward due to the
notion of ordering of a univariate sample, which can readily reveal
possible outliers. In the bivariate case, however, it is often difficult
to detect extreme points and verify whether a sample of residuals is a
reasonable realisation from a fitted model. We propose a new framework,
implemented as the bivrp R package, available on the Comprehensive R
Archive Network. Our framework uses the same principles of the
simulation envelope in a half-normal plot, but as a simulation polygon
for each point in a bivariate sample. By using algorithms of convex hull
construction and polygon area reduction, we describe how our method
works and illustrate its functionality with examples using simulated
bivariate normal data and real bivariate count data. We show how
different model diagnostics can produce different results and pinpoint
potential drawbacks of our approach, such as the limitations in terms of
computational burden and convex hull bias.

\clearpage

%-----------------------------------------------------------------------
% 04/10
%-----------------------------------------------------------------------
\addcontentsline{toc}{section}{(04/10)
  Modelagem de capturas em peso inflacionadas de zeros no Baixo Rio
  Amazonas}

\begin{center}
  {\footnotesize 04/10/2018 (Quinta-feira)}\\
  \textbf{\large
    Modelagem de capturas em peso inflacionadas de zeros no Baixo Rio
    Amazonas}\\
  Júlio César Pereira\footnote{UFSCar/Sorocaba, contato:
    \url{julio-pereira@ufscar.br}}
\end{center}

Neste trabalho desenvolvemos um modelo hierárquico bayesiano em três
estágios para acomodar a inflação zero nas capturas resultantes da pesca
comercial no Baixo Rio Amazonas. Inicialmente, modelamos o número de
viagens de pesca (N), dado que N> 0 modelamos a probabilidade de sucesso
na captura de certas espécies e finalmente, na terceira etapa, o peso
capturado foi modelado.

\clearpage

%-----------------------------------------------------------------------
% 11/10
%-----------------------------------------------------------------------
\addcontentsline{toc}{section}{(11/10)
  New two-stage sampling designs based on neoteric ranked set sampling}

\begin{center}
  {\footnotesize 11/10/2018 (Quinta-feira)}\\
  \textbf{\large
    New two-stage sampling designs based on neoteric ranked set
    sampling}\\
  César Augusto Taconelli\footnote{UFPR/LEG, contato:
    \url{taconeli@ufpr.br}}
\end{center}

Neoteric ranked set sampling (NRSS) is a recently developed sampling
plan, derived from the well-known ranked set sampling (RSS) scheme. In
this work, we propose and evaluate the performance of five different
alternatives of two-stage sampling designs based on NRSS. These
approaches configure alternatives to some other two-stage sampling
designs based on RSS. We conducted an extensive Monte Carlo simulation
study comprising, besides the proposed estimators, RSS, NRSS and the
usual double ranked set sampling scheme (DRSS). The simulated results
indicated that the new two-stage NRSS designs outperform their RSS-based
counterparts, providing estimators for the population mean with lower
mean square error. An application with data of the diameter and height
of pine trees complements the study.

\clearpage

%-----------------------------------------------------------------------
% 18/10
%-----------------------------------------------------------------------
\addcontentsline{toc}{section}{(18/10)
  Métodos de Pesquisa e Elaboração de Questionários}

\begin{center}
  {\footnotesize 18/10/2018 (Quinta-feira)}\\
  \textbf{\large
  Métodos de Pesquisa e Elaboração de Questionários}\\
  Denise Britz do N. Silva\footnote{ENCE/IBGE, contato:
    \url{denisebritz@gmail.com}}
\end{center}

Várias das informações utilizadas na gestão pública, no desenvolvimento
da ciência e no exercício da cidadania são provenientes de pesquisas
amostrais ou censos. A necessidade de estatísticas de boa qualidade
resulta na constante atenção aos métodos estatísticos para produção e
análise da informação. Nesta palestra, apresenta-se uma introdução aos
principais conceitos e métodos para o planejamento e a execução de
pesquisas, bem como para a elaboração de instrumentos de
coleta. Destaca-se, também, a importância do contínuo desenvolvimento de
novos métodos estatísticos na área de pesquisas quantitativas.

\clearpage

\end{document}
